\documentclass{article}
\usepackage{titling}
\title{Practical Runtime Verification of Performance in event based systems using Metric Temporal Logic}

\author{George Fakidis}

\begin{document}
\maketitle
\begin{abstract}
	Event Based Systems(EBS) are the foundation of modern cloud system infrastructure. Event Driven Architecture(EDA) is a scalable architectural pattern that enables operations at an unprecedented scale. Performance considerations are of utmost importance to ensure cost efficiency and achieving QoS guarantees. In this paper we formalise performance [profiling] of events, in a manner that allows us to reuse existing methodologies and algorithms. We show that performance monitoring of series of events can be reduced to a problem of checking the temporal constraints of the events under study using event logs. We also create a preliminary tool to integrate this runtime verification into existing technology stacks.
\end{abstract}


\section{Introduction}
Software Systems have grown considerably in the last years, and are now used in critical sectors like 1,2,3.
Modern large scale systems serve millions of users and execute billions of operations per second.
To enable computation at this unprecedented scale, sophisticated scalable architectures are required.
One of the most popular and widely adopted is Event-Driven Architecture that centers its attention to messages sent between different systems in order to complete the desired computations.
Therefore it is necessary to reason about the performance and correctness software and systems in a manner that can provide guarantees and be relied upon. Event-based systems are complex due to the flexible,unspecified and asynchronous interactions they enable and thus require expressive specification techniques that can be checked at runtime.
\par
[This will be written from Wikipedia and mb a couple foundational papers] Formal Methods are a set of mathematical tools that enable us to reason about software in a complete, sound and verifiable manner using specifications that can be checked rigorously. Formal Methods that pursue completeness can sometimes be unapplicable in certain cases such as 1,2,3. Runtime verification uses data generated by a running system(traces) in order to detect violations of expressed properties and tries to provide guarantees for the system being studied.

\par
A bit more detail of what event-based systems are, how do they look in practice(w. diagrams and a simple example).


\par
Performance in Event-Based Systems can be more precisely expressed as:
\begin{itemize}
	\item Throughput
	\item Latency
\end{itemize}

\par
A brief explanation of Metric Temporal Logic(MTL) and some visual examples and the intuition behind its operators.


\section{Related Work}
\begin{itemize}
	\item How is Performance Measured in Event-Based/Event-Driven Systems?
	\item How are QoS guarantees made in such systems (at a model level)?
	\item What formal Methods are used in such tasks?
\end{itemize}

Formal Methods used in other works.
\begin{itemize}
	\item UPAAL
	\item PRISM
	\item Markov Chains (of Discrete or Continuous time)
	\item Timed Temporal Logics (TTL $\rightarrow$ TCTL,MTL)
	\item Probabilistic Temporal Logics (PCTL, PLTL)
	\item Petri nets
\end{itemize}

\section{Formalising Event-Based Systems}

\section{Formal expressions in MTL}

\section{Integration of verification procecss into running EBS}


\section{Tool Implementation}
Architecture Diagram, a bit of discussion about the technology choice and some details.
\section{Case Study}
This is where a simple case study will be done, regarding cloud resources and QoS guarantees.



\end{document}
